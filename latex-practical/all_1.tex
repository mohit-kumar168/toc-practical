\documentclass[a4paper,12pt]{report}
\usepackage[a4paper, margin=1in]{geometry}
\usepackage{graphicx}
\usepackage{amsmath}
\usepackage{booktabs}
\usepackage{setspace}
\usepackage{hyperref}
\usepackage{makeidx}
\makeindex
\onehalfspacing
\title{Economic Analysis using \LaTeX}
\author{Abhishek Sahu}
\date{\today}
\begin{document}
\maketitle
\tableofcontents
\chapter{Introduction}
This mini project demonstrates how to combine multiple \LaTeX{} features | including l
\section{Objectives}
\begin{itemize}
\item Learn basic \LaTeX{} document structure.
\item Apply formatting, lists, and mathematical symbols.
\item Present data neatly in tabular form.
\end{itemize}
\chapter{Data Analysis}
\section{Table Example}
\begin{table}[h!]
\centering
6\caption{Sample Economic Data}
\begin{tabular}{l c c}
\toprule
Year & GDP Growth (\%) & Inflation (\%) \\
\midrule
2020 & -7.3 & 6.2 \\
2021 & 8.9 & 5.5 \\
2022 & 6.8 & 6.0 \\
\bottomrule
\end{tabular}
\end{table}
\section{Mathematical Equation}
The GDP growth rate ($g$) can be calculated as:
\[
g = \frac{GDP_t - GDP_{t-1}}{GDP_{t-1}} \times 100
\]
For example, if GDP in 2021 was 210 and in 2020 was 200:
\[
g = \frac{210 - 200}{200} \times 100 = 5\%
\]
\section{Figure Example}
\begin{figure}[h!]
\centering
\includegraphics[width=0.5\textwidth]{image.png}
\caption{Example Chart Showing GDP Trend}
\end{figure}
\chapter{Conclusion}
Using \LaTeX, we can design neat, consistent, and professional academic reports that c
\begin{thebibliography}{9}
\bibitem{kottwitz2021}
Stefan Kottwitz,
\textit{LaTeX Beginner’s Guide (2nd ed.)},
Packet Publishing Ltd., 2021.
\bibitem{lamport1994}
Leslie Lamport,
\textit{LaTeX: A Document Preparation System},
Pearson Education, 1994.
\end{thebibliography}
\end{document}
