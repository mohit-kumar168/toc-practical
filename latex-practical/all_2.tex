\documentclass[a4paper,12pt]{article}
\usepackage[a4paper, margin=1in]{geometry}
\usepackage{amsmath, booktabs, graphicx, devanagari, setspace}
\onehalfspacing
\title{Integration of Mathematics and Indian Knowledge Systems}
\author{Abhishek Sahu}
\date{\today}
\begin{document}
\maketitle
\section*{Abstract}
This document demonstrates integration of mathematical concepts and Indian heritage us
\section{Mathematical Concept}
The quadratic formula is given by:
\[
x = \frac{-b \pm \sqrt{b^2 - 4ac}}{2a}
\]
\section{Key Indian Scholars}
\begin{itemize}
\item \dn{} (Aryabhata)
\item \dn{} (Bhaskaracharya)
\item \dn{} (Chanakya)
\end{itemize}
\section{Data Representation}
\begin{table}[h!]
\centering
\caption{Contributions of Ancient Scholars}
\begin{tabular}{l l}
\toprule
2Scholar & Contribution \\
\midrule
Aryabhata & Concept of Zero \\
Bhaskaracharya & Differential Calculus \\
Chanakya & Arthashastra (Economics) \\
\bottomrule
\end{tabular}
\end{table}
\section{Conclusion}
The harmony between mathematics and ancient Indian wisdom proves that knowledge evolve
\end{document}
